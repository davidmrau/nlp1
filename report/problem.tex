\section{Problem Analysis}
\label{problem}

- How does the language model decide between plural or singular form in sentences like ``The products that the company [produce/produces]'' or ``The products of the company [are/is]''?

- The model needs to decide the dependency relations between verbs and nouns,
and it has to decide the count of nouns and verbs.
We assume that the model is able to `learn' the count of a noun or verb (may fail for some nouns as discussed in Linzen). We focus this discussion on how the model may infer dependencies. 

- syntactic clues. Example: the word `that' indicates the beginning of a relative clause with a noun and a verb, therefore the noun 'companies' is associated with the verb `produce' in the first sentence. In contrast, the word `of' in the second sentence indicates a possessive relation with a noun and does not expect a verb, therefore the noun company is not associated to the verb `are' in the second sentence.

- semantic clues. Example: the noun 'companies' is associated with the verb 'produce' in the first sentence because this combination of words (company/companies ... produces/produce) appears a lot in the training sentences
\\
\textit{Hypothesis: the model uses both semantic and syntactic clues to determine which
form of the verb is most likely (singular/plural).
}
\\
EXPERIMENT IDEA 1:
We investigate whether the model (primarily) uses semantic or syntactic clues to determine the singular/plural form of the verb. 
We do this by

1. observing how the model behaves on sentences in which semantic and syntactic clues are consistent with each other, e.g. ``The products that the company [produce/produces]''

2. observing how the model behaves on sentences that are semantically nonsense, e.g. ``The products that the company [swim/swims]''

3. observing how the model behaves on sentences in which semantic and syntactic clues contradict each other, e.g. ``The companies that the product [produce/produces]''. 

We generate these sentences using nouns and verbs that occur frequently in the corpus (lots of semantic info) and for nouns and verbs that occur 
infrequently (not so much semantic info). We compare the results of all six combinations.

EXPERIMENT IDEA 2:
(future work?) A different but related question is whether
syntax (function words) alone provides enough information 
to decide the plurality of verbs.
We investigate this question (for the english language). 
First we 'strip the semantic clues' from the sentences by replacing all nouns and verbs with their pos tag, e.g ``The NNS that the NN [VBP/VBZ]''. Then we perform two experiments: 
2a) we ask humans to decide the plurality of the verbs for a set of test sentences
in this form. Can humans decide the plurality of a verb without knowing the meaning of the verb and the nouns?
2b) we train a model on these kind of processed sentences,
we can train a language model and/or a more specialized model trained with the target
to decide the plurality of verbs.
We then evaluate the model(s) on a test set in the stripped form. 
Can the model learn to decide the plurality of a verb based on syntactic clues, without knowing the meaning of
the verbs and the nouns?
(remark: by removing the 'semantic noise' we make it much easier for the model to learn noun-verb dependencies based on syntax.)

%
% File naaclhlt2018.tex
%
%% Based on the style files for NAACL-HLT 2018, which were
%% Based on the style files for ACL-2015, with some improvements
%%  taken from the NAACL-2016 style
%% Based on the style files for ACL-2014, which were, in turn,
%% based on ACL-2013, ACL-2012, ACL-2011, ACL-2010, ACL-IJCNLP-2009,
%% EACL-2009, IJCNLP-2008...
%% Based on the style files for EACL 2006 by 
%%e.agirre@ehu.es or Sergi.Balari@uab.es
%% and that of ACL 08 by Joakim Nivre and Noah Smith

\documentclass[11pt,a4paper]{article}
\usepackage[hyperref]{naaclhlt2018}
\usepackage{times}
\usepackage{latexsym}
\usepackage{graphics, graphicx}
\usepackage{url}
\usepackage{todonotes}
\usepackage{amsmath}
\aclfinalcopy % Uncomment this line for the final submission
%\def\aclpaperid{***} %  Enter the acl Paper ID here

%\setlength\titlebox{5cm}
% You can expand the titlebox if you need extra space
% to show all the authors. Please do not make the titlebox
% smaller than 5cm (the original size); we will check this
% in the camera-ready version and ask you to change it back.

\newcommand\BibTeX{B{\sc ib}\TeX}

\title{Empirical Evaluation of LSTM Language Models \bigskip
on Syntactic Dependencies}



\author{
Maartje de Jonge\\
\tt Student ID: 194107\\
{\tt maartjedejonge@gmail.com} \\\And
David Rau \\
\tt Student ID: 17725184\\
{\tt david.rau@student.uva.nl} \\}


\date{}

\begin{document}
\maketitle

\begin{abstract}

\noindent 

LSTM models have become increasingly popular for
the task of language modeling, mostly
because of their capability to capture long-distance
dependencies. 
Dependencies in natural language are often sensitive
to syntactic structure;
capturing such dependencies is challenging for LSTM models 
since they do not explicitly incorporate syntactic structure.
In this paper we focus on noun-verb number agreement
as an example of a syntactic dependency.
We investigate the effect of
statistic, syntactic and semantic information
on the number prediction results of LSTM language models.
Our results show that the model is able
to learn the number for most nouns and verbs
using statistic regularities,
we also found evidence for a modest sensitivity
to syntactic structure. 
The main contribution of this paper is that it provides insights
into what information LSTM language models actually
use to determine the plurality number of a predicted verb.
\end{abstract}

% main idea
%- language model, number agreement

% key findings
%- able to learn the numbers of nouns and verbs
%- modest sensitivity to syntactic structure


%----------------------------------------------------------------------------------------
%	INTRODUCTION
%----------------------------------------------------------------------------------------
\section{Introduction}
- motivation

- problem area

- problem itself

- research question

- approach

\paragraph{Outline}
The remainder of this report is organized as follows \ldots

=====================
RNN
LSTM
language modeling, NLP

sequence models
no explicit syntactic structure
can RNNs learn syntactic dependencies from a natural corpus?

Linzen paper investigates number agreement as an example
compares explicitly trained models and more general language model
conclusion: explicit supervision required to learn syntactic dependencies.


While the Linzen paper focuses on explicitly
trained models, we further investigate a language model
in more detail to see what information it
actually uses to decide on the number of a predicted verb.
We take an empirical approach;
treat the model as a black box
and learn about it
by observing its behavior
in carefully designed experiments.

We first repeat the experiments from the linzen paper
on a language model trained on PTB corpus.
We reach similar conclusions,
TODO: what about distance?
While the model can establish number agreement
for simple cases without intervening nouns,
it fails for more complex cases scoring below average.

We then take a closer look on the errors
it made on simple cases without intervening nouns.
By analyzing noun/verb combinations
we identify verbs that show a strong preference for either
one of the forms (singlar or plural),
ignoring the count of the noun.
In addition, we identified nouns that
the model seems to have failed to learn or even mis-learned
the number.
We explained these cases by frequency statistics on the training corpus.

We then look at syntactic information exposed 
by function words such as 'that' and 'of'.
We measure the performance of the model on 
generated sentences constructed following a specific
syntactic structure. 
From this experiment we learned that
function words can help the model to establish
number agreement with the structurally relevant noun.
But the evidence is very week.

Thus far we looked at random generated nonsense
sentences, ignoring the fact that sentences 
are typically about something.
In real world sentences some verbs 
tend to form subject verb dependencies with certain nouns
while other nouns do not mae sense.
This can help the model.
The price of the products stabilize/stabilizes.
prices are semanticly related to stabilize,
while products are not.
We compare prefixes with a noun that
is semantically related 
compare with same prefic with noun replaced by random noun.
to prefixes with a noun that is not.
Does the semantic relation helps to establish
number agreement with the relevant noun?

From these experiments we conclude that
...


%----------------------------------------------------------------------------------------
%	BACKGROUND
%----------------------------------------------------------------------------------------



%A background section, in which you explain what is language modelling,
%what kind of language model you studied, and what is the problem you
%focus on. To write this background section, you can use the suggested
%papers, but also add papers;

%----------------------------------------------------------------------------------------
%	PREVIOUS WORK
%----------------------------------------------------------------------------------------

\section{Related Work}
\label{related work}

%summary of ~\citep{Linzen2016}

%Linzen 2016
Our work builds on the results described in ~\citep{Linzen2016}. In this section 
we summarize this paper. 

%%%% Research Question
LSTMs are sequence models that can capture long distance statistical regularities,
but do not have built-in hierarchical representations.
In ~\citep{Linzen2016} Linzen et al. investigate whether LSTMs are able to capture
dependencies that follow from syntactic structure. More specific,
the paper investigates number agreement in English subject-verb dependencies
as an example of a structure sensitive dependency.


%LSTMs
%- can capture long distance statistical regularities
%- do not explicitly incorporate syntactic structure
%- question: can LSTM capture dependencies that follow from syntactic structure, from a %corpus without syntactic annotations?
%- more specific: investigate number agreement in English subject-verb dependencies
%as an example of a structure sensitive dependency

%%%% Experiments
The paper compares the performance of LSTMs
trained with an explicit grammatical target as training objective,
as well as a more generic language model trained with the target to
predict the next word. 
The models are trained on a corpus without syntactic annotations (Wikipedia).
The models are evaluated on real sentences taken from this corpus
that were sampled based on their grammatical complexity. 

%- train models with different training objectives
%  - with explicit grammatical target as training objective
%    - number prediction
%    - verb infliction
%    - grammatical judgement
%  - a more generic language model with the objective to predict the next word
%- corpus: wikipedia

%- evaluate how these models perform on simple and more complex sentences:
%  - simple cases: noun that is the subject closeby verb, no intervening nouns
%  - effect of long distance: lot of words inbetween noun and verb
%  - complex cases: intervening nouns with different number inbetween head of subject noun %and verb
  
%%%% Results
All models achieved an overall error rate below 7\%. However,
the explicitly trained models perform much better (0.8\% - 2,5\%)
compared to the language model (6.8\%). 
The overall high accuracy for all models can be explained by
the fact that most naturally occurring sentences are actually
simple (no intervening nouns between verb and subject).

The differences between the models become more pronounced,
when evaluating them on grammatically complex sentences.
While the performance of the grammatically trained models
degrades slowly, achieving error rates below 20\% even with
four intervening nouns of opposite number;
the language model at the other hand does worse-than-chance
on most complex cases. The worse-than-chance performance
indicates that the intevening nouns actively confuse
the model. 

%%%%%%%%% Language model

- high or reasonable overall accuracy for all models, explained by the fact that most real world sentences are actually simple (no attractors)
   - NP, VI, GJ, LM
- all explicitly trained language models perform reasonable on complex cases.
   - distance
   - attractors
- language model performs bad on complex cases. sensitive to most recent noun.
  - more complex objective, lack of data? no, google also fails.

%%%% Detailed analysis of results
More detailed error analysis shows that
- function words are important, by comparing with a baseline model trained on noun verb sequences
- relative clauses are challenging, especially when relativizer misses
- some errors in identifying nouns (due to ambiguity: drives) and identifying the number of a noun

- activations: units that track main subject, subject of current clause, embedding status, number of main clause subject/most recent noun

%%%% Conclusion
- LSTM can capture grammatical structure given targeted supervision
- language modeling is insufficient for capturing syntax sensitive dependencies
- authors advice: supplement language modeling objectives with more explicit targets
for tasks in which it is desirable to capture syntactic dependencies.


%----------------------------------------------------------------------------------------
%	EXPERIMENTS
%----------------------------------------------------------------------------------------

\section{Replication}
\label{replication}

In this section we replicate some experiments of~\citep{Linzen2016} to get a general impression of how well our model is able to establish number agreement.
All experiments in this paper are performed using an LSTM model\footnote{
\url{https://github.com/pytorch/examples/tree/master/word_language_model}
} trained on a general language modeling task.

\subsection{Singular and Plural Nouns}

\textbf{Data:} 
The language model was tested on lower case sentences that were generated from the Wall Street Journal section of the Penn Treebank \citep{Marcus1993}. Therefore, 40 nouns and verbs, that were amongst the most common and occur in the test corpus as well as in the language model's corpus, were extracted. Those words build the base for the sentence generation for the subsequent experiments. 

\textbf{Model evaluation:} In order to evaluate the performance of our model, we query it with sentences containing both, the plural (\ref{sent:wrong}) and the singular (\ref{sent:right}) mode of the verb: 

\begin{equation}
	\label{sent:wrong}
	\textnormal{the producer plan}
\end{equation}
\begin{equation}
	\label{sent:right}
	\textnormal{the producer  \textbf{plans}}
\end{equation}
Following the experiments in \citep{Linzen2016} we examine the models error rate predicting the number of a verb. That is, the model receives the words leading up to the verb and needs to decide between the singular and plural forms of a particular verb.
%We examine sentences with a given number of nouns of opposite number intervening %between the subject and the verb. 

    \begin{figure}
    \centering
        \includegraphics[scale=0.5]{2b.pdf}
            	\caption{Error rates of the language model plotted against: presence and number of last intervening noun}
        \label{fig:2b}
    \end{figure}
    ~ %add desired spacing between images, e. g. ~, \quad, \qquad, \hfill etc. 
      %(or a blank line to force the subfigure onto a new line)
    \begin{figure}
    \centering
        \includegraphics[scale=0.5]{2c.pdf}
        \label{fig:2c}
            \caption{Error rates of the language model plotted against count of attractors in dependencies with homogeneous intervention.}
    \end{figure}

We first test how the model's ability to predict the number of the verb is affected by none and one intervening noun, respectively. If there is an intervening noun we keep track whether the number of the noun differs from the number of the subject. If it does so it is referred to as an \textit{agreement attractor}. In this way we can easily spot whether the model makes use of the most obvious heuristic: choosing the the number of the verb only in dependence of the last intervening noun in the sentence.

As depicted in Figure \ref{fig:2b} our model performs slightly worse for plural subjects (17.7\% error rate) than for singular (15,2\% error rate) when no intervening nouns are present. An intervening noun with the same number as the subject causes a slight increase of the error rate to 18.6\% and a decrease to 15.1\%, respectively. However, when the number of the subject differs from the intervening noun the error rates increased dramatically; in singular subjects to 57,9\% in plural subjects to 60,8\%. The fact that it performs worse than predicting the number by chance implies that the model indeed predicts the number of the verb in dependence of the last noun and therefore fails to find the dependency between verb and noun.

\subsection{Multiple Attractors}

In the following we examine the error rate when adding multiple attractors to the sentence. In order to avoid the model of being distracted by an intervening noun with the same number as the subject we only insert nouns with the same number. Linzen et al.~\citep{Linzen2016} refer to such as  \textit{ dependencies with homogeneous intervention}. Consider the following example sentences:
\begin{equation}
	\label{sent:right2}
	\textnormal{the \textbf{interest} in the \underline{shares} of the \underline{businesses} \textbf{rises} \dots}
\end{equation}
\begin{equation}
	\label{sent:wrong2}
	\textnormal{the \textbf{interest} in the \underline{shares} of the \textit{business} \textbf{rises} \dots}
\end{equation}
In (\ref{sent:right2}) the underlined represent homogeneous interventions, whereas in (\ref{sent:wrong2}) the number of the intervening nouns differ. The bold words highlight the dependency between noun and corresponding verb.

Figure \ref{fig:2c} shows that with an increasing number of noun interventions the error rate goes from 20.8\% (0 attractors) up to 73.6\% (4 attractors) which is worse than randomly guessing the number of the verb. 

%TODO: describe 2b_least in text and give implications

\section{Own Experiments}
\label{own-experiments}


\end{multicols}

\begin{figure}
    \centering
    \begin{subfigure}[b]{0.4\textwidth}
    \caption{}
        \includegraphics[width=\textwidth]{2b_least.pdf}
        \label{fig:2b_least}
    \end{subfigure}
    ~ %add desired spacing between images, e. g. ~, \quad, \qquad, \hfill etc. 
      %(or a blank line to force the subfigure onto a new line)
    \begin{subfigure}[b]{0.4\textwidth}

    \end{subfigure}
    \caption{\ref{fig:2b_least} shows the error rates of the language model for the least frequent nouns in the corpus as the subject and when no intervening nouns are present}
\end{figure}

\begin{multicols}{2}


To predict the correct number of a given verb,
the language model should be able to
1. identify the noun that is the head of the subject for the verb
2. establish the number of the noun (non-trivial since no knowledge of -s postfix for plurals)
3. establish the number of the given verb forms (also non-trivial).

In the first sub section we investigate if our model is able to
do this for simple cases with only a single noun in the prefix.
%
In the second sub section we investigate if our model can handle
more complex cases with two nouns in the prefix,
and what information it then uses to identify the head of the subject.

\subsection{Noun-Verb Agreement in Simple Cases}


In this section we investigate the ability of the model to
establish number agreement for nouns and verbs in the simplest case,
following the pattern: ``The <noun> <verb>''. Notice that
the determiner ``The'' clearly indicates the position of the noun.

1. TEST DATA: 
Generate 100 x 50 simple prefixes in singular and plural form, like ``the company ... [produce/produces]'' 
and ``the companies ... [produce/produces]''.
The sentences do not need to be meaningful.
(randomly pick the nouns and verbs without taking into account their frequency in training data)

2. EXPERIMENT: 
Evaluate how the language model performs on these sentences 
and conclude whether or not the model is sensitive to noun-verb agreement in simple cases. 
(i.e. if it tends to predict plural verbs for plural nouns and singular verbs for singular nouns)
output 1: cross table with correct vs predicted
output 2: overall error rate number, error rate number for plurals, error rate number for singulars 


2. FURTHER ANALYSIS:
- Build a matrix: Nouns x Verbs, 
the entries tell whether the model predicted singular(1)
or plural(0).

- Sort the columns based on their sum
- Sort the rows for singular nouns based on their sum (upper half)
- Sort the rows for plural nouns based on their sum (lower half)
- print as an image (black = plural prediction, white is singular prediction)

- Discuss the picture:
  - Do we see verbs that clearly prefer singular resp. plural?
      (what is their plurality ratio?)
  - Do we see nouns that clearly prefer singular resp. plural
      (i.e. the model established their plurality), 
    or nouns that more or less follow the preference of the verbs?
     what is their frequency (count) in training corpus?)
  - Do we see nouns for which the model thinks
    that they are plural while they are in fact singular?
    
- Optional 1:
  - also include picture which shown models uncertainty in grey teints
    (uncertainty = evaluate(produces)/(evaluate(produce) + evaluate(produces))
    this is instead of max(evaluate(produces), evaluate(produce))

- Optional 2:
  - print noun counts as an y-axis
  - print verb plurality rates as an x-axis
  (both are one dimensional matrices)
  (helps visualize these characteristics)
  (expectation: plurality rates decrease from left to right)
  (expectation: nouns in the middle appear less frequent in training corpus)
  (expectation: more plurality in upper half, resp. singularity in lower half. That is higher error rate)

- Optional 3:
  - repeat experiment for least frequent nouns
  (Is ordering of columns more or less the same?)
  (does the pattern looks different now?)
  
  
- Optional 4:
(probably not in paper)
Show that pattern is not caused by random variations,
i.e. random pattern looks different


%hypothesis: the model falls back to pure frequencies of the two verb forms
%in case it failed to learn the counts because of sparsity in the data.

%Further inspect the error cases, why did it fail for these very simple sentences:
%a) maybe the model failed to learn the number of the noun (i.e. low occurrence in training data for the given noun form)?
%b) maybe the model failed to learn the number of the verb forms (i.e. low occurrence in training data for both verb forms)?
%c) maybe the occurence in the training data of the incorrect but predicted verb form is much higher than the occurrence 
%in training data of the correct form?
%output a: histogram with x-axis z-score noun, y axis count (or percentage) of verbs that fall in this range
%output b: histogram with x-axis z-score verb (max of both), y axis count (or percentage) of verbs that fall in this range
%output c: histogram with x-axis percentage of predicted form, y axis count (or percentage) of verbs that fall in this range
%output(?): a scatter plot, x-axis z-score noun vs y-axis z-score verb, color gradient is percentage of predicted verb form

 
\subsection{Noun-Verb Agreement in Complex Cases}

In Section \ref{replication} we analysed the performance of the model
on complex sentences, containing one or more 
nouns.
The results show that the model is very
sensitive to the most recent noun,
performing worse-than-chance with only one single attractor
(Figure \ref{fig:2c}). 

In this section we investigate whether
syntactic and semantic information
can still help the model 
to establish number agreement
in case of multiple nouns.
We focus on sentences with exactly two nouns
of opposite number.


\subsubsection{Syntactic Information}

%%%%% OBJECTIVE
Function words such as 'that' or 'of' carry 
important information about the syntactic structure of a sentence.
In this Section we investigate if the model
uses this information to establish number agreement
for complex sentences.

%%%% TEST DATA
We generated sets of sentence prefixes using 
different syntactic templates.
An example is:
"The [Noun1] of the [Noun2] ... [VBZ/VBP]".
We instantiate the templates by randomly
picking two nouns and a verb from a set of frequently
used nouns and verbs. 
Each combination of nouns and verbs instantiate
two prefixes that differ by their plurality.
For example:
"The company of the governments ...[know/knows]"
"The companies of the government ...[know/knows]".
%Notice that these prefixes are typically not semantically
%meaningful since the nouns and verbs are randomly picked.

We generated 2 x 1000 sentences per template,
for a total of 11 templates.
The sentences for each template are constructed using the same
noun, verb combinations.
We defined 7 templates for which the first noun is the head of the subject,
while 4 templates have the second noun as the head of the subject.
The templates are shown in figure \ref{x}, respectively figure \ref{y}.

%%%% EXPERIMENT:
We evaluate how the model responds to the generated test inputs.
That is, for each test prefix we let the model decide between 
the singular and plural form of the given verb. 
We measure the error rate for each template.
However, instead of showing the error rates we
show how much the language model tends to agree with the most recent noun.
This correcponds to the error rate for the templates in \ref{x},
while it corresponds to accuracy for the templates in \ref{y}.
Showing the `last noun agreement rate' makes it easier
to compare the behavior of the model for different templates.

%%%% ANALYSIS:
The results are shown in Figure \ref{z},
using green and red colors to indicate if 
the last noun is actually the head of the subject (green)
or not (red). 
%
We see that all bars are above the 0.5 rate,
which shows that the model is most
likely to agree with the last noun,
even in cases where this is syntactically incorrect. 
%
We also see that the red bars are slightly
lower than the green bars on average.
This indicates that the model still has some sensitivity
to syntactic information that points in the direction 
of the first noun as the head of the subject.
%

%
We further discuss two special cases,
namely T1 and T11.
T1 "the \_ and the \_ " is special because it 
actually contains two singular nouns,
instead of one singular and one plural noun. 
The predicted verb should be plural because of the
conjunction word "and".
The two nouns of opposite number make it even harder for the model
to establish number agreement. 
This may explain the high error rate
for T1 (Figure \ref{s}, first red bar).
%
In T11 we used different templates for the singular case
(the \_ 's \_ ) and the plural case (the \_' \_).
The average result is shown in 
Figure \ref{s}, last green bar.
We suspect that the accuracy is relatively
low in this case because the plural possessive form
may not occur frequently. 
Indeed, a closer inspection of the numbers showed that
the singular template had an accuracy
of ..., while the accuracy of the plural template
was considerably lower, ....
%

%%%% DISCUSSION / CONCLUSION:
We conclude that, although the model 
performs bad on complex sentences it
still has some sensitivity for syntactic 
structure exposed by function words. 


\includegraphics[scale=0.5]{screenshot-syntactic-templates} 
%TODO: save picture instead of making screenshot 
%TODO title Templates below
 
 
\begin{tabular}{ l l r }
  T1    & the \_ and the \_     &  0.77 \\
  T2    & the \_ in the \_      &  0.59 \\
  T3    & the \_ by the \_      &  0.69 \\
  T4    & the \_ of the \_      &  0.61 \\
  T5    & the \_ near the \_    &  0.63\\
  T6    & the \_ at the \_      &  0.70\\
  T7    & the \_ without the \_ & 0.67  \\
\end{tabular}

\begin{tabular}{ l l r }
  T8    & the \_ the \_         &  0.78\\
  T9    & the \_ that the \_    &  0.79\\
  T10   & the \_ whether the \_ &  0.85\\
  T11   & the \_ 's \_ (for plural: the \_' \_)          &  0.72 \\
\end{tabular}

\subsubsection{Semantic Information}

We now look at semantic clues, i.e. companies
are more likely to produce than employees.
We investigate if the model is able to establish
number agreement between the verb and the semantic related noun,
ignoring the non-semantic related attractor noun.
  
1. TEST DATA:
We construct a testset (100+ sentences) consisting of pairs with singular and plural prefixes, in the following format  
``The NN of the NNS ...[VBZ/VBP]'' and
``The NNS of the NN ...[VBZ/VBP]'' 
whereby the head of the subject (the first noun)
is semantically related to the verb, while the attractor (the second noun)
is randomly picked. An example is:
``The prices of the employee ...[stabilizes/stabilize]'' and 
``The price of the employees ...[stabilizes/stabilize]''.

In addition we construct a comparison set consisting of the same prefixes
(same verb and attractor noun),
but with the difference that the first noun is also randomly picked and
therefore most likely not semantically related. Example:
``The newspapers of the employee ...[stabilizes/stabilize]'' and 
``The newspaper of the employees ...[stabilizes/stabilize]''.


Remark: the semantically related noun/verb combinations can either be manually
chosen, or preferable, can be learned from the data by looking at verb noun combinations that 'frequently occur together'. 
'frequently occur together' means: count(VBP + NNS)/count(VBP) is 
relatively high.

EXPERIMENT:
Evaluate our model on the constructed testset and on the comparison set.
If the model scores significantly better on the testset
then it shows sensitivity to semantic clues.

%(remark: we can repeat the experiment with the nouns interchanged and check that we %score worse now,
%e.g. "The employee that the prices ... [stabilize/stabilizes]" 
%In that case syntax and semantics are inconsistent)



%----------------------------------------------------------------------------------------
%	CONCLUSION
%----------------------------------------------------------------------------------------
\section{Conclusion and Future Work}
\label{conclusion}


%\todo[prepend]{This is a subtle one.
%I think Linzen considers simple cases not to be proof of capturing 
%grammatical structure,
%since they can be solved by simple last noun agreement. However,
%I basically agree with you that learning counts and recognizing nouns
%and verbs are also grammatical achievements. 
%In addition, Linzen is negative over general LSTM language models 
%because he contrasts those with his explicitly trained grammatical LSTM models.
%Anyway, I would probably not refer to Linzen here in this way.
%}
Overall, we conclude that LSTM language models
are able to capture grammatical properties to some extend.
Our model was able to predict the correct number of noun/verb pairs 
for most of the simple sentences;
from which we conclude that it does encode the plurality number of those words. 
Through further analysis of the models decisions in the simplest cases, we could find some evidence that the model occasionally falls back to statistical properties of the training corpus, such as word frequencies.
This may happen for example when the ratio between the plural 
and the singular form of a verb is biased.

On more complex cases, e.g. sentences containing intervening nouns 
of opposite number
in between the subject and the verb, 
the model most often fails to establish number agreement. 
In this case we could observe that the model at least shows 
some sensitivity for syntactic information.
That is, the model is most likely to agree with the
most recent noun, but it is a bit less likely to do so 
in case this is syntactically incorrect.

For future work it would be interesting to perform similar experiments on real world sentences rather than on artificially constructed ones. 
The sentences that we generated are typically not semantically meaningful,
in addition, they are syntactically less divers than real world sentences. 
It remains an open question whether our model would perform differently on those.

An interesting aspect of analysing LSTMs is to look into the embeddings and further investigate the internal state of the network rather than treating it like a black box. Looking at the activations of the LSTMs might give further insights into the strenghts and weaknesses of the model and could lead to a better error analysis then by looking solely at the predictions. 

To follow up on the experiments it would also be interesting to train the model with explicit syntactic structures instead of relying too much on statistical properties of the training corpus.





% include your own bib file like this:
%\bibliographystyle{acl}
%\bibliography{naaclhlt2018}
\bibliography{article}
\bibliographystyle{aaai-named}

\end{document}

%%%%%%%%%%%%%%%%%%%%%%%%%%%%%%%%%%%%%%%%%
% Journal Article
% LaTeX Template
% Version 1.3 (9/9/13)
%
% This template has been downloaded from:
% http://www.LaTeXTemplates.com
%
% Original author:
% Frits Wenneker (http://www.howtotex.com)
%
% License:
% CC BY-NC-SA 3.0 (http://creativecommons.org/licenses/by-nc-sa/3.0/)
%
%%%%%%%%%%%%%%%%%%%%%%%%%%%%%%%%%%%%%%%%%

%----------------------------------------------------------------------------------------
%	PACKAGES AND OTHER DOCUMENT CONFIGURATIONS
%----------------------------------------------------------------------------------------

\documentclass[twoside]{article}
\usepackage{graphics, graphicx}
\usepackage{lipsum} % Package to generate dummy text throughout this template
\usepackage[sc]{mathpazo} % Use the Palatino font
\usepackage[T1]{fontenc} % Use 8-bit encoding that has 256 glyphs
\linespread{1.05} % Line spacing - Palatino needs more space between lines
\usepackage{microtype} % Slightly tweak font spacing for aesthetics

\usepackage[hmarginratio=1:1,top=32mm,columnsep=20pt]{geometry} % Document margins
\usepackage{multicol} % Used for the two-column layout of the document
\usepackage[hang, small,labelfont=bf,up,textfont=it,up]{caption} % Custom captions under/above floats in tables or figures
\usepackage{booktabs} % Horizontal rules in tables
\usepackage{float} % Required for tables and figures in the multi-column environment - they need to be placed in specific locations with the [H] (e.g. \begin{table}[H])
\usepackage{hyperref} % For hyperlinks in the PDF
\usepackage{color, colortbl}

\bibliographystyle{plainnat}
\usepackage{natbib}
\usepackage{lettrine} % The lettrine is the first enlarged letter at the beginning of the text
\usepackage{paralist} % Used for the compactitem environment which makes bullet points with less space between them
\usepackage{subcaption}
\usepackage{abstract} % Allows abstract customization
\renewcommand{\abstractnamefont}{\normalfont\bfseries} % Set the "Abstract" text to bold
\renewcommand{\abstracttextfont}{\normalfont\small\itshape} % Set the abstract itself to small italic text

\usepackage{titlesec} % Allows customization of titles
\renewcommand\thesection{\Roman{section}} % Roman numerals for the sections
\renewcommand\thesubsection{\Roman{subsection}} % Roman numerals for subsections
\titleformat{\section}[block]{\large\scshape\centering}{\thesection.}{1em}{} % Change the look of the section titles
\titleformat{\subsection}[block]{\large}{\thesubsection.}{1em}{} % Change the look of the section titles

\usepackage{fancyhdr} % Headers and footers
\pagestyle{fancy} % All pages have headers and footers
\fancyhead{} % Blank out the default header
\fancyfoot{} % Blank out the default footer
\fancyfoot[RO,LE]{\thepage} % Custom footer text

%----------------------------------------------------------------------------------------
%	TITLE SECTION
%----------------------------------------------------------------------------------------

\title{\vspace{-15mm}\fontsize{24pt}{10pt}\selectfont\textbf{Finding Grammatical Structure}} % Article title


\author{
\large
\textsc{David Rau (11725148), Jer\^ome Mutgeert (....)},Maartje de Jonge (.....) \\[2mm] % Your name
\normalsize University of Amsterdam \\ % Your institution
\normalsize \href{mailto:david.rau@student.uva.nl, ...,...}{david.rau@student.uva.nl, ..., ...} % Your email address
\vspace{-5mm}
}
\date{}


%----------------------------------------------------------------------------------------

\begin{document}

\maketitle % Insert title

\thispagestyle{fancy} % All pages have headers and footers

%----------------------------------------------------------------------------------------
%	ABSTRACT
%----------------------------------------------------------------------------------------

\begin{abstract}

\noindent 
- main idea
- key findings

\end{abstract}

%----------------------------------------------------------------------------------------
%	INTRODUCTION
%----------------------------------------------------------------------------------------

\begin{multicols}{2} % Two-column layout throughout the main article text

\section{Introduction}
- motivation

- problem area

- problem itself

- research question

- approach

\paragraph{Outline}
The remainder of this report is organized as follows \ldots


%----------------------------------------------------------------------------------------
%	BACKGROUND
%----------------------------------------------------------------------------------------

\section{Background}
\label{background}

- what is language modeling?

- what kind of language modelling did we study?

- what is the problem that we focus on?


%----------------------------------------------------------------------------------------
%	PREVIOUS WORK
%----------------------------------------------------------------------------------------

\section{Related Work}
\label{related work}

%summary of ~\citep{Linzen2016}

%Linzen 2016
Our work builds on the results described in ~\citep{Linzen2016}. In this section 
we summarize this paper. 

%%%% Research Question
LSTMs are sequence models that can capture long distance statistical regularities,
but do not have built-in hierarchical representations.
In ~\citep{Linzen2016} Linzen et al. investigate whether LSTMs are able to capture
dependencies that follow from syntactic structure. More specific,
the paper investigates number agreement in English subject-verb dependencies
as an example of a structure sensitive dependency.


%LSTMs
%- can capture long distance statistical regularities
%- do not explicitly incorporate syntactic structure
%- question: can LSTM capture dependencies that follow from syntactic structure, from a %corpus without syntactic annotations?
%- more specific: investigate number agreement in English subject-verb dependencies
%as an example of a structure sensitive dependency

%%%% Experiments
The paper compares the performance of LSTMs
trained with an explicit grammatical target as training objective,
as well as a more generic language model trained with the target to
predict the next word. 
The models are trained on a corpus without syntactic annotations (Wikipedia).
The models are evaluated on real sentences taken from this corpus
that were sampled based on their grammatical complexity. 

%- train models with different training objectives
%  - with explicit grammatical target as training objective
%    - number prediction
%    - verb infliction
%    - grammatical judgement
%  - a more generic language model with the objective to predict the next word
%- corpus: wikipedia

%- evaluate how these models perform on simple and more complex sentences:
%  - simple cases: noun that is the subject closeby verb, no intervening nouns
%  - effect of long distance: lot of words inbetween noun and verb
%  - complex cases: intervening nouns with different number inbetween head of subject noun %and verb
  
%%%% Results
All models achieved an overall error rate below 7\%. However,
the explicitly trained models perform much better (0.8\% - 2,5\%)
compared to the language model (6.8\%). 
The overall high accuracy for all models can be explained by
the fact that most naturally occurring sentences are actually
simple (no intervening nouns between verb and subject).

The differences between the models become more pronounced,
when evaluating them on grammatically complex sentences.
While the performance of the grammatically trained models
degrades slowly, achieving error rates below 20\% even with
four intervening nouns of opposite number;
the language model at the other hand does worse-than-chance
on most complex cases. The worse-than-chance performance
indicates that the intevening nouns actively confuse
the model. 

%%%%%%%%% Language model

- high or reasonable overall accuracy for all models, explained by the fact that most real world sentences are actually simple (no attractors)
   - NP, VI, GJ, LM
- all explicitly trained language models perform reasonable on complex cases.
   - distance
   - attractors
- language model performs bad on complex cases. sensitive to most recent noun.
  - more complex objective, lack of data? no, google also fails.

%%%% Detailed analysis of results
More detailed error analysis shows that
- function words are important, by comparing with a baseline model trained on noun verb sequences
- relative clauses are challenging, especially when relativizer misses
- some errors in identifying nouns (due to ambiguity: drives) and identifying the number of a noun

- activations: units that track main subject, subject of current clause, embedding status, number of main clause subject/most recent noun

%%%% Conclusion
- LSTM can capture grammatical structure given targeted supervision
- language modeling is insufficient for capturing syntax sensitive dependencies
- authors advice: supplement language modeling objectives with more explicit targets
for tasks in which it is desirable to capture syntactic dependencies.


\section{Problem Analysis}
\label{problem}

- How does the language model decide between plural or singular form in sentences like ``The products that the company [produce/produces]'' or ``The products of the company [are/is]''?

- The model needs to decide the dependency relations between verbs and nouns,
and it has to decide the count of nouns and verbs.
We assume that the model is able to `learn' the count of a noun or verb (may fail for some nouns as discussed in Linzen). We focus this discussion on how the model may infer dependencies. 

- syntactic clues. Example: the word `that' indicates the beginning of a relative clause with a noun and a verb, therefore the noun 'companies' is associated with the verb `produce' in the first sentence. In contrast, the word `of' in the second sentence indicates a possessive relation with a noun and does not expect a verb, therefore the noun company is not associated to the verb `are' in the second sentence.

- semantic clues. Example: the noun 'companies' is associated with the verb 'produce' in the first sentence because this combination of words (company/companies ... produces/produce) appears a lot in the training sentences
\\
\textit{Hypothesis: the model uses both semantic and syntactic clues to determine which
form of the verb is most likely (singular/plural).
}
\\
EXPERIMENT IDEA 1:
We investigate whether the model (primarily) uses semantic or syntactic clues to determine the singular/plural form of the verb. 
We do this by

1. observing how the model behaves on sentences in which semantic and syntactic clues are consistent with each other, e.g. ``The products that the company [produce/produces]''

2. observing how the model behaves on sentences that are semantically nonsense, e.g. ``The products that the company [swim/swims]''

3. observing how the model behaves on sentences in which semantic and syntactic clues contradict each other, e.g. ``The companies that the product [produce/produces]''. 

We generate these sentences using nouns and verbs that occur frequently in the corpus (lots of semantic info) and for nouns and verbs that occur 
infrequently (not so much semantic info). We compare the results of all six combinations.

EXPERIMENT IDEA 2:
(future work?) A different but related question is whether
syntax (function words) alone provides enough information 
to decide the plurality of verbs.
We investigate this question (for the english language). 
First we 'strip the semantic clues' from the sentences by replacing all nouns and verbs with their pos tag, e.g ``The NNS that the NN [VBP/VBZ]''. Then we perform two experiments: 
2a) we ask humans to decide the plurality of the verbs for a set of test sentences
in this form. Can humans decide the plurality of a verb without knowing the meaning of the verb and the nouns?
2b) we train a model on these kind of processed sentences with the target
to decide the plurality of the verb.
We then ask the model to complete the sentences of a test set in the stripped form. 
Can the model learn to decide the plurality without knowing the meaning of
the verb and the nouns?


%----------------------------------------------------------------------------------------
%	EXPERIMENTS
%----------------------------------------------------------------------------------------

\section{Replication}
\label{replication}

\subsection{Singular and Plural Nouns}

\textbf{Data:} 
The language model was tested on lower case sentences that were generated from the Wall Street Journal section of the Penn Treebank \citep{Marcus1993}. Therefore, 40 nouns and verbs, that were amongst the most common and occur in the test corpus as well as in language model's corpus, were extracted. Those words build the base for the sentence generation for the subsequent experiments. 

\textbf{Model evaluation:} In order to evaluate the performance of our model, we query it with the sentence containing both, the plural (\ref{sent:wrong}) and the singular (\ref{sent:right}) mode of the verb: 

\begin{equation}
	\label{sent:wrong}
	\textnormal{the producer plan}
\end{equation}
\begin{equation}
	\label{sent:right}
	\textnormal{the producer  \textbf{plans}}
\end{equation}
According to the experiments in \citep{Linzen2016} we examine the models error rate predicting the number of a verb in a sentence when nouns intervene between the subject and the verb. 

We next test how the model's ability to predict the number of the verb is affected by none and one intervening noun, respectively. If there is an intervening noun we keep track whether the number of the noun differs from the number of the subject. If it does so it is referred to as an\textit{agreement attractor}. In this way we can easily spot whether the model makes use of the most obvious heuristic: choosing the the number of the verb only in dependence of the last intervening noun in the sentence.
\end{multicols}

\begin{figure}
    \centering
    \begin{subfigure}[b]{0.4\textwidth}
    \caption{}
        \includegraphics[width=\textwidth]{2b.pdf}
        \label{fig:2b}
    \end{subfigure}
    ~ %add desired spacing between images, e. g. ~, \quad, \qquad, \hfill etc. 
      %(or a blank line to force the subfigure onto a new line)
    \begin{subfigure}[b]{0.4\textwidth}
    	\caption{}
        \includegraphics[width=\textwidth]{2c.pdf}
        \label{fig:2c}
    \end{subfigure}
    \caption{Error rates of the language model plotted against: \ref{fig:2b} presence and number of last intervening noun; \ref{fig:2c} count of attractors in dependencies with homogeneous intervention}
\end{figure}

\begin{multicols}{2}
As depicted in Figure \ref{fig:2b} our model performs slightly better for plural subjects (17.7\% error rate) than for singular (15,2\% error rate) when no intervening nouns are present. An intervening noun with the same number as the subject causes a slight increase of the error rate to 18.6\% and a decrease to 15.1\%, respectively. Surprisingly, when the number of the subject differs from the intervening noun the error rates increased dramatically; in singular subjects to 57,9\% in plural subjects to 60,8\%. The fact that it performs worse than predicting the number by chance implies that the model indeed predicts the number of the verb in dependence of the last noun and therefore fails to find the dependency between verb and noun.

\subsection{Multiple Attractors}

In the following we examine the error rate when adding multiple attractors to the sentence. In order to avoid the model of being distracted by an intervening noun with same number as the subject we only insert nouns with the same number . \citep{Linzen2016} refer to such as  \textit{ dependencies with homogeneous intervention}. In (\ref{sent:right2}) the underlined represent homogeneous interventions, whereas in (\ref{sent:wrong2}) the number of the intervening nouns differ. The bold words highlight the dependency between noun and corresponding verb.

\begin{equation}
	\label{sent:right2}
	\textnormal{the \textbf{interest} in the \underline{shares} of the \underline{businesses} \textbf{rises} \dots}
\end{equation}
\begin{equation}
	\label{sent:wrong2}
	\textnormal{the \textbf{interest} in the \underline{shares} of the \textit{business} \textbf{rises} \dots}
\end{equation}
Figure \ref{fig:2c} shows that with an increasing number of noun interventions the error rate goes from 20.8\% (0 attractors) up to 73.6\% (4 attractors) which is close to randomly guessing the number of the verb. 

\subsection{Distance}

The ability of the model to track the number of a noun may degrade
when the noun and the verb are far apart.
Below we investigate the effect of distance on number agreement for simple,
single noun, cases.

1. TEST DATA:
Pick the simple cases and generate multiple additional cases for each by adding 0 to n additional words (using a template so that the sentence prefix is at least grammatically correct)
[preferable: use real sentences of different length without attractors for this]

2. EXPERIMENT:
Evaluate how the language model performs on these sentences 
and conclude whether or not the model is robust to distance. 
(i.e. if it tends to predict plural verb for plural nouns even when they are far apart)
output: line diagram with x-axis distance (number of words inbetween), y-axis error rate


\section{Own Experiments}
\label{own-experiments}


\end{multicols}

\begin{figure}
    \centering
    \begin{subfigure}[b]{0.4\textwidth}
    \caption{}
        \includegraphics[width=\textwidth]{2b_least.pdf}
        \label{fig:2b_least}
    \end{subfigure}
    ~ %add desired spacing between images, e. g. ~, \quad, \qquad, \hfill etc. 
      %(or a blank line to force the subfigure onto a new line)
    \begin{subfigure}[b]{0.4\textwidth}

    \end{subfigure}
    \caption{\ref{fig:2b_least} shows the error rates of the language model for the least frequent nouns in the corpus as the subject and when no intervening nouns are present}
\end{figure}

\begin{multicols}{2}


To predict the correct number of a given verb,
the language model should be able to
1. identify the noun that is the head of the subject for the verb
2. establish the number of the noun (non-trivial since no knowledge of -s postfix for plurals)
3. establish the number of the given verb forms (also non-trivial).

In the first sub section we investigate if our model is able to
do this for simple cases with only a single noun in the prefix.
%
In the second sub section we investigate if our model can handle
more complex cases with two nouns in the prefix,
and what information it then uses to identify the head of the subject.

\subsection{Noun-Verb Agreement in Simple Cases}


In this section we investigate the ability of the model to
establish number agreement for nouns and verbs in the simplest case,
following the pattern: ``The <noun> <verb>''. Notice that
the determiner ``The'' clearly indicates the position of the noun.

1. TEST DATA: 
Generate 100 x 50 simple prefixes in singular and plural form, like ``the company ... [produce/produces]'' 
and ``the companies ... [produce/produces]''.
The sentences do not need to be meaningful.
(randomly pick the nouns and verbs without taking into account their frequency in training data)

2. EXPERIMENT: 
Evaluate how the language model performs on these sentences 
and conclude whether or not the model is sensitive to noun-verb agreement in simple cases. 
(i.e. if it tends to predict plural verbs for plural nouns and singular verbs for singular nouns)
output 1: cross table with correct vs predicted
output 2: overall error rate number, error rate number for plurals, error rate number for singulars 


2. FURTHER ANALYSIS:
- Build a matrix: Nouns x Verbs, 
the entries tell whether the model predicted singular(1)
or plural(0).

- Sort the columns based on their sum
- Sort the rows for singular nouns based on their sum (upper half)
- Sort the rows for plural nouns based on their sum (lower half)
- print as an image (black = plural prediction, white is singular prediction)

- Discuss the picture:
  - Do we see verbs that clearly prefer singular resp. plural?
      (what is their plurality ratio?)
  - Do we see nouns that clearly prefer singular resp. plural
      (i.e. the model established their plurality), 
    or nouns that more or less follow the preference of the verbs?
     what is their frequency (count) in training corpus?)
  - Do we see nouns for which the model thinks
    that they are plural while they are in fact singular?
    
- Optional 1:
  - also include picture which shown models uncertainty in grey teints
    (uncertainty = evaluate(produces)/(evaluate(produce) + evaluate(produces))
    this is instead of max(evaluate(produces), evaluate(produce))

- Optional 2:
  - print noun counts as an y-axis
  - print verb plurality rates as an x-axis
  (both are one dimensional matrices)
  (helps visualize these characteristics)
  (expectation: plurality rates decrease from left to right)
  (expectation: nouns in the middle appear less frequent in training corpus)
  (expectation: more plurality in upper half, resp. singularity in lower half. That is higher error rate)

- Optional 3:
  - repeat experiment for least frequent nouns
  (Is ordering of columns more or less the same?)
  (does the pattern looks different now?)
  
  
- Optional 4:
(probably not in paper)
Show that pattern is not caused by random variations,
i.e. random pattern looks different


%hypothesis: the model falls back to pure frequencies of the two verb forms
%in case it failed to learn the counts because of sparsity in the data.

%Further inspect the error cases, why did it fail for these very simple sentences:
%a) maybe the model failed to learn the number of the noun (i.e. low occurrence in training data for the given noun form)?
%b) maybe the model failed to learn the number of the verb forms (i.e. low occurrence in training data for both verb forms)?
%c) maybe the occurence in the training data of the incorrect but predicted verb form is much higher than the occurrence 
%in training data of the correct form?
%output a: histogram with x-axis z-score noun, y axis count (or percentage) of verbs that fall in this range
%output b: histogram with x-axis z-score verb (max of both), y axis count (or percentage) of verbs that fall in this range
%output c: histogram with x-axis percentage of predicted form, y axis count (or percentage) of verbs that fall in this range
%output(?): a scatter plot, x-axis z-score noun vs y-axis z-score verb, color gradient is percentage of predicted verb form

 
\subsection{Noun-Verb Agreement in Complex Cases}

In Section \ref{replication} we analysed the performance of the model
on complex sentences, containing one or more 
nouns.
The results show that the model is very
sensitive to the most recent noun,
performing worse-than-chance with only one single attractor
(Figure \ref{fig:2c}). 

In this section we investigate whether
syntactic and semantic information
can still help the model 
to establish number agreement
in case of multiple nouns.
We focus on sentences with exactly two nouns
of opposite number.


\subsubsection{Syntactic Information}

%%%%% OBJECTIVE
Function words such as 'that' or 'of' carry 
important information about the syntactic structure of a sentence.
In this Section we investigate if the model
uses this information to establish number agreement
for complex sentences.

%%%% TEST DATA
We generated sets of sentence prefixes using 
different syntactic templates.
An example is:
"The [Noun1] of the [Noun2] ... [VBZ/VBP]".
We instantiate the templates by randomly
picking two nouns and a verb from a set of frequently
used nouns and verbs. 
Each combination of nouns and verbs instantiate
two prefixes that differ by their plurality.
For example:
"The company of the governments ...[know/knows]"
"The companies of the government ...[know/knows]".
%Notice that these prefixes are typically not semantically
%meaningful since the nouns and verbs are randomly picked.

We generated 2 x 1000 sentences per template,
for a total of 11 templates.
The sentences for each template are constructed using the same
noun, verb combinations.
We defined 7 templates for which the first noun is the head of the subject,
while 4 templates have the second noun as the head of the subject.
The templates are shown in figure \ref{x}, respectively figure \ref{y}.

%%%% EXPERIMENT:
We evaluate how the model responds to the generated test inputs.
That is, for each test prefix we let the model decide between 
the singular and plural form of the given verb. 
We measure the error rate for each template.
However, instead of showing the error rates we
show how much the language model tends to agree with the most recent noun.
This correcponds to the error rate for the templates in \ref{x},
while it corresponds to accuracy for the templates in \ref{y}.
Showing the `last noun agreement rate' makes it easier
to compare the behavior of the model for different templates.

%%%% ANALYSIS:
The results are shown in Figure \ref{z},
using green and red colors to indicate if 
the last noun is actually the head of the subject (green)
or not (red). 
%
We see that all bars are above the 0.5 rate,
which shows that the model is most
likely to agree with the last noun,
even in cases where this is syntactically incorrect. 
%
We also see that the red bars are slightly
lower than the green bars on average.
This indicates that the model still has some sensitivity
to syntactic information that points in the direction 
of the first noun as the head of the subject.
%

%
We further discuss two special cases,
namely T1 and T11.
T1 "the \_ and the \_ " is special because it 
actually contains two singular nouns,
instead of one singular and one plural noun. 
The predicted verb should be plural because of the
conjunction word "and".
The two nouns of opposite number make it even harder for the model
to establish number agreement. 
This may explain the high error rate
for T1 (Figure \ref{s}, first red bar).
%
In T11 we used different templates for the singular case
(the \_ 's \_ ) and the plural case (the \_' \_).
The average result is shown in 
Figure \ref{s}, last green bar.
We suspect that the accuracy is relatively
low in this case because the plural possessive form
may not occur frequently. 
Indeed, a closer inspection of the numbers showed that
the singular template had an accuracy
of ..., while the accuracy of the plural template
was considerably lower, ....
%

%%%% DISCUSSION / CONCLUSION:
We conclude that, although the model 
performs bad on complex sentences it
still has some sensitivity for syntactic 
structure exposed by function words. 


\includegraphics[scale=0.5]{screenshot-syntactic-templates} 
%TODO: save picture instead of making screenshot 
%TODO title Templates below
 
 
\begin{tabular}{ l l r }
  T1    & the \_ and the \_     &  0.77 \\
  T2    & the \_ in the \_      &  0.59 \\
  T3    & the \_ by the \_      &  0.69 \\
  T4    & the \_ of the \_      &  0.61 \\
  T5    & the \_ near the \_    &  0.63\\
  T6    & the \_ at the \_      &  0.70\\
  T7    & the \_ without the \_ & 0.67  \\
\end{tabular}

\begin{tabular}{ l l r }
  T8    & the \_ the \_         &  0.78\\
  T9    & the \_ that the \_    &  0.79\\
  T10   & the \_ whether the \_ &  0.85\\
  T11   & the \_ 's \_ (for plural: the \_' \_)          &  0.72 \\
\end{tabular}

\subsubsection{Semantic Information}

We now look at semantic clues, i.e. companies
are more likely to produce than employees.
We investigate if the model is able to establish
number agreement between the verb and the semantic related noun,
ignoring the non-semantic related attractor noun.
  
1. TEST DATA:
We construct a testset (100+ sentences) consisting of pairs with singular and plural prefixes, in the following format  
``The NN of the NNS ...[VBZ/VBP]'' and
``The NNS of the NN ...[VBZ/VBP]'' 
whereby the head of the subject (the first noun)
is semantically related to the verb, while the attractor (the second noun)
is randomly picked. An example is:
``The prices of the employee ...[stabilizes/stabilize]'' and 
``The price of the employees ...[stabilizes/stabilize]''.

In addition we construct a comparison set consisting of the same prefixes
(same verb and attractor noun),
but with the difference that the first noun is also randomly picked and
therefore most likely not semantically related. Example:
``The newspapers of the employee ...[stabilizes/stabilize]'' and 
``The newspaper of the employees ...[stabilizes/stabilize]''.


Remark: the semantically related noun/verb combinations can either be manually
chosen, or preferable, can be learned from the data by looking at verb noun combinations that 'frequently occur together'. 
'frequently occur together' means: count(VBP + NNS)/count(VBP) is 
relatively high.

EXPERIMENT:
Evaluate our model on the constructed testset and on the comparison set.
If the model scores significantly better on the testset
then it shows sensitivity to semantic clues.

%(remark: we can repeat the experiment with the nouns interchanged and check that we %score worse now,
%e.g. "The employee that the prices ... [stabilize/stabilizes]" 
%In that case syntax and semantics are inconsistent)


%----------------------------------------------------------------------------------------
%	DISCUSSION
%----------------------------------------------------------------------------------------
\section{Discussion and Future Work}
\label{discussion}

Future work: 
- replicate Linzen experiment with real world sentences
* less artificial
* semantically meaningful
* syntactically more divers
Does the model perform better or worse?

%----------------------------------------------------------------------------------------
%	CONCLUSION
%----------------------------------------------------------------------------------------
\section{Conclusion}
\label{conclusion}

- Conclusion \ldots


%----------------------------------------------------------------------------------------
%	REFERENCE LIST
%----------------------------------------------------------------------------------------

\bibliographystyle{abbrv}
\bibliography{article}


%----------------------------------------------------------------------------------------

\end{multicols}

\paragraph{Contributions}

\begin{itemize}
  \item \textbf{Maartje de Jonge}
  \begin{itemize}
    \item Abstract
    \item Introduction
    \item Related Work
    \item Problem Analysis
    \item Own Experiments
    \begin{itemize}
       \item Simple Cases: Design, Implementation
       \item Syntactic Information: Design, Implementation, Text
       \item Semantic Information: Design
    \end{itemize}        
  \end{itemize}

  \item \textbf{Jerome Mutgeert}
   \begin{itemize}
      \item Replication
      \begin{itemize}
         \item Distances
      \end{itemize}
    \item Own Experiments
    \begin{itemize}
       \item Semantic Information: Implementation, Text
    \end{itemize}        
    \item Discussion and Future Work
    \item Conclusion
   \end{itemize}

  \item \textbf{David Rau}
    \begin{itemize}
      \item Background
      \item Replication
      \begin{itemize}
         \item Last Intervening Noun
         \item Attractor Counts
      \end{itemize}
    \item Own Experiments
    \begin{itemize}
       \item Simple Cases: Implementation, Text
    \end{itemize}        
  \end{itemize}

\end{itemize}




\end{document}

%%%%%%%%%%%%%%%%%%%%%%%%%%%%%%%%%%%%%%%%%
% Journal Article
% LaTeX Template
% Version 1.3 (9/9/13)
%
% This template has been downloaded from:
% http://www.LaTeXTemplates.com
%
% Original author:
% Frits Wenneker (http://www.howtotex.com)
%
% License:
% CC BY-NC-SA 3.0 (http://creativecommons.org/licenses/by-nc-sa/3.0/)
%
%%%%%%%%%%%%%%%%%%%%%%%%%%%%%%%%%%%%%%%%%

%----------------------------------------------------------------------------------------
%	PACKAGES AND OTHER DOCUMENT CONFIGURATIONS
%----------------------------------------------------------------------------------------

\documentclass[twoside]{article}

\usepackage{lipsum} % Package to generate dummy text throughout this template

\usepackage[sc]{mathpazo} % Use the Palatino font
\usepackage[T1]{fontenc} % Use 8-bit encoding that has 256 glyphs
\linespread{1.05} % Line spacing - Palatino needs more space between lines
\usepackage{microtype} % Slightly tweak font spacing for aesthetics

\usepackage[hmarginratio=1:1,top=32mm,columnsep=20pt]{geometry} % Document margins
\usepackage{multicol} % Used for the two-column layout of the document
\usepackage[hang, small,labelfont=bf,up,textfont=it,up]{caption} % Custom captions under/above floats in tables or figures
\usepackage{booktabs} % Horizontal rules in tables
\usepackage{float} % Required for tables and figures in the multi-column environment - they need to be placed in specific locations with the [H] (e.g. \begin{table}[H])
\usepackage{hyperref} % For hyperlinks in the PDF

\usepackage{lettrine} % The lettrine is the first enlarged letter at the beginning of the text
\usepackage{paralist} % Used for the compactitem environment which makes bullet points with less space between them

\usepackage{abstract} % Allows abstract customization
\renewcommand{\abstractnamefont}{\normalfont\bfseries} % Set the "Abstract" text to bold
\renewcommand{\abstracttextfont}{\normalfont\small\itshape} % Set the abstract itself to small italic text

\usepackage{titlesec} % Allows customization of titles
\renewcommand\thesection{\Roman{section}} % Roman numerals for the sections
\renewcommand\thesubsection{\Roman{subsection}} % Roman numerals for subsections
\titleformat{\section}[block]{\large\scshape\centering}{\thesection.}{1em}{} % Change the look of the section titles
\titleformat{\subsection}[block]{\large}{\thesubsection.}{1em}{} % Change the look of the section titles

\usepackage{fancyhdr} % Headers and footers
\pagestyle{fancy} % All pages have headers and footers
\fancyhead{} % Blank out the default header
\fancyfoot{} % Blank out the default footer
\fancyhead[C]{Running title } % Custom header text
\fancyfoot[RO,LE]{\thepage} % Custom footer text

%----------------------------------------------------------------------------------------
%	TITLE SECTION
%----------------------------------------------------------------------------------------

\title{\vspace{-15mm}\fontsize{24pt}{10pt}\selectfont\textbf{Finding Grammatical Structure}} % Article title


\author{
\large
\textsc{Maartje de Jonge} %\thanks{A thank you or further information}\\[2mm] % Your name
\normalsize University of Amsterdam \\ % Your institution
%\normalsize \href{mailto:john@smith.com}{john@smith.com} % Your email address
%\normalsize 194107 % Your email address
\vspace{-5mm}
\and
\large
\textsc{David Rau} %\thanks{A thank you or further information}\\[2mm] % Your name
\normalsize University of Amsterdam \\ % Your institution
%\normalsize \href{mailto:john@smith.com}{john@smith.com} % Your email address
%\normalsize <st nr> % Your email address
\and
\vspace{-5mm}
\large
\textsc{Jer\^ome Mutgeert} %\thanks{A thank you or further information}\\[2mm] % Your name
\normalsize University of Amsterdam \\ % Your institution
%\normalsize \href{mailto:john@smith.com}{john@smith.com} % Your email address
%\normalsize <st nr> % Your email address
%\vspace{-5mm}
}



\date{}

%----------------------------------------------------------------------------------------

\begin{document}

\maketitle % Insert title

\thispagestyle{fancy} % All pages have headers and footers

%----------------------------------------------------------------------------------------
%	ABSTRACT
%----------------------------------------------------------------------------------------

\begin{abstract}

\noindent 
- main idea
- key findings

\end{abstract}

%----------------------------------------------------------------------------------------
%	INTRODUCTION
%----------------------------------------------------------------------------------------

\begin{multicols}{2} % Two-column layout throughout the main article text

\section{Introduction}
- motivation

- problem area

- problem itself

- research question

- approach

\paragraph{Outline}
The remainder of this report is organized as follows \ldots


%----------------------------------------------------------------------------------------
%	BACKGROUND
%----------------------------------------------------------------------------------------

\section{Background}
\label{background}

- what is language modeling?

- what kind of language modelling did we study?

- what is the problem that we focus on?


%----------------------------------------------------------------------------------------
%	PREVIOUS WORK
%----------------------------------------------------------------------------------------


\section{Previous Work}
\label{previous work}

- summary of ~\cite{Linzen2016}

\section{Problem Analysis}
\label{problem}

- How does the language model decide between plural or singular form in sentences like ``The products that the company [produce/produces]'' or ``The products of the company [are/is]''?

- The model needs to decide the dependency relations between verbs and nouns,
and it has to decide the count of nouns and verbs.
We assume that the model is able to `learn' the count of a noun or verb (may fail for some nouns as discussed in Linzen). We focus this discussion on how the model may infer dependencies. 

- syntactic clues. Example: the word `that' indicates the beginning of a relative clause with a noun and a verb, therefore the noun 'companies' is associated with the verb `produce' in the first sentence. In contrast, the word `of' in the second sentence indicates a possessive relation with a noun and does not expect a verb, therefore the noun company is not associated to the verb `are' in the second sentence.

- semantic clues. Example: the noun 'companies' is associated with the verb 'produce' in the first sentence because this combination of words (company/companies ... produces/produce) appears a lot in the training sentences
\\
\textit{Hypothesis: the model uses both semantic and syntactic clues to determine which
form of the verb is most likely (singular/plural).
}
\\
EXPERIMENT IDEA 1:
We investigate whether the model (primarily) uses semantic or syntactic clues to determine the singular/plural form of the verb. 
We do this by

1. observing how the model behaves on sentences in which semantic and syntactic clues are consistent with each other, e.g. ``The products that the company [produce/produces]''

2. observing how the model behaves on sentences that are semantically nonsense, e.g. ``The products that the company [swim/swims]''

3. observing how the model behaves on sentences in which semantic and syntactic clues contradict each other, e.g. ``The companies that the product [produce/produces]''. 

We generate these sentences using nouns and verbs that occur frequently in the corpus (lots of semantic info) and for nouns and verbs that occur 
infrequently (not so much semantic info). We compare the results of all six combinations.

EXPERIMENT IDEA 2:
(future work?) A different but related question is whether
syntax (function words) alone provides enough information 
to decide the plurality of verbs.
We investigate this question (for the english language). 
First we 'strip the semantic clues' from the sentences by replacing all nouns and verbs with their pos tag, e.g ``The NNS that the NN [VBP/VBZ]''. Then we perform two experiments: 
2a) we ask humans to decide the plurality of the verbs for a set of test sentences
in this form. Can humans decide the plurality of a verb without knowing the meaning of the verb and the nouns?
2b) we train a model on these kind of processed sentences with the target
to decide the plurality of the verb.
We then ask the model to complete the sentences of a test set in the stripped form. 
Can the model learn to decide the plurality without knowing the meaning of
the verb and the nouns?


%----------------------------------------------------------------------------------------
%	EXPERIMENTS
%----------------------------------------------------------------------------------------

\section{Experiments}
\label{experiments}

- Our model:

\subsection{Replication Linzen 2016}
\label{replication}

- We replicate some experiments from ~\cite{Linzen2016} with our own language model \ldots

\subsection{Own Experiments}
\label{own-experiments}

- We investigate the following question about grammatical structure \ldots

%----------------------------------------------------------------------------------------
%	CONCLUSION
%----------------------------------------------------------------------------------------
\section{Conclusion}
\label{conclusion}

- Conclusion \ldots

%----------------------------------------------------------------------------------------
%	DISCUSSION
%----------------------------------------------------------------------------------------
\section{Discussion}



%----------------------------------------------------------------------------------------
%	REFERENCE LIST
%----------------------------------------------------------------------------------------

\bibliographystyle{abbrv}
\bibliography{article}


%----------------------------------------------------------------------------------------

\end{multicols}

\end{document}
